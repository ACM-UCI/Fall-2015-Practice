\normalfont\documentclass[letterpaper,11pt]{article}
\usepackage{amsmath, amsfonts,amssymb,latexsym}
\usepackage{fullpage}
\usepackage{parskip}
\usepackage{graphicx}

\begin{document}
\section*{Problem 1: Alena's Schedule}
(Taken from: http://codeforces.com/problemset/problem/586/A) \newline
Alena has successfully passed the entrance exams to the university and is now looking forward to start studying.

One two-hour lesson at the Russian university is traditionally called a pair, it lasts for two academic hours (an academic hour is equal to 45 minutes).

The University works in such a way that every day it holds exactly $n$ lessons. Depending on the schedule of a particular group of students, on a given day, some pairs may actually contain classes, but some may be empty (such pairs are called breaks).

The official website of the university has already published the schedule for tomorrow for Alena's group. Thus, for each of the $n$ pairs she knows if she has  a class at that time or not.

Alena's house is far from the university, so if there are breaks, she doesn't always go home. Alena has time to go home only if the break consists of at least two free pairs in a row, otherwise she waits for the next pair at the university.

Of course, Alena does not want to be sleepy during pairs, so she will sleep as long as possible, and will only come to the first pair that is presented in her schedule. Similarly, if there are no more pairs, then Alena immediately goes home.

Alena appreciates the time spent at home, so she always goes home when it is possible, and returns to the university only at the beginning of the next pair. Help Alena determine for how many pairs she will stay at the university. Note that during some pairs Alena may be at the university waiting for the upcoming pair.

\textbf{Input} \newline
The first line of the input contains a positive integer $n$ $(1 \leq n \leq 100)$ — the number of lessons at the university. The second line contains $n$ numbers $a_i$ $(0 \leq a_i \leq 1)$, separated by spaces. The value $a_i$ is 1 if Alena has that pair, and 0 if she is free.

\textbf{Output} \newline
Print a single number — the number of pairs during which Alena stays at the university.

\textbf{Examples}
\begin{itemize}
\item \textbf{Input} \newline
5 \newline
0 1 0 1 1

\textbf{Output} \newline
4

\item \textbf{Input} \newline
7 \newline
1 0 1 0 0 1 0

\textbf{Output} \newline
4
\end{itemize}

\newpage

\section*{Problem 2: Laurenty and Shop}
(Taken from: http://codeforces.com/problemset/problem/586/B)

A little boy named Laurenty has been playing his favourite game Nota for quite a while and is now very hungry. The boy wants to make sausage and cheese sandwiches, but first, he needs to buy a sausage and some cheese.

The town where Laurenty lives in is not large. The houses in it are located in two rows, with $n$ houses in each row. Laurenty lives in the very last house of the second row. The only shop in town is placed in the first house of the first row.

The first and second rows are separated with the main avenue of the city. The adjacent houses of one row are separated by streets. Each crosswalk of a street or an avenue has some traffic lights. In order to cross the street, you need to press a button on the traffic light, wait for a while for the green light and cross the street. Different traffic lights can have different waiting time.

The traffic light on the crosswalk from the $j$-th house of the $i$-th row to the $(j + 1)$-th house of the same row has waiting time equal to $a_{ij}$ $(1 \leq i \leq 2, 1 \leq j \leq n - 1)$. For the traffic light on the crossing from the $j$-th house of one row to the $j$-th house of another row the waiting time equals $b_j$ $(1 \leq j \leq n)$. The city doesn't have any other crossings.

The boy wants to get to the store, buy the products and go back. The main avenue of the city is wide enough, so the boy wants to cross it exactly once on the way to the store and exactly once on the way back home. The boy would get bored if he had to walk the same way again, so he wants the way home to be different from the way to the store in at least one crossing.

Help Laurenty determine the minimum total time he needs to wait at the crossroads.

\textbf{Input} \newline
The first line of the input contains a single integer $n$ $(2 \leq n \leq 50)$ — the number of houses in each row.

Each of the next two lines contains $n - 1$ space-separated integers — values $a_{ij}$ $(1 \leq a_{ij} \leq 100)$.

The last line contains $n$ space-separated integers $b_j$ $(1 \leq j \leq n)$.

\textbf{Output} \newline
Print a single integer — the least total time Laurenty needs to wait at the crossroads, given that he crosses the avenue only once both on his way to the store and on his way back home.

\textbf{Examples}
\begin{itemize}
\item \textbf{Input} \newline
4 \newline
1 2 3 \newline
3 2 1 \newline
3 2 2 3

\textbf{Output} \newline
12

\end{itemize}

\newpage

\section*{Problem 3: Philip and Trains}
(Taken from: http://codeforces.com/problemset/problem/585/B) \newline
Philip is located in one end of the tunnel and wants to get out on the other side. The tunnel is a rectangular field consisting of three rows and $n$ columns. At the beginning of the game the hero is in some cell of the leftmost column. Some number of trains ride towards the hero. Each train consists of two or more neighbouring cells in some row of the field.

All trains are moving from right to left at a speed of two cells per second, and the hero runs from left to right at the speed of one cell per second. For simplicity, the game is implemented so that the hero and the trains move in turns. First, the hero moves one cell to the right, then one square up or down, or stays idle. Then all the trains move twice simultaneously one cell to the left. Thus, in one move, Philip definitely makes a move to the right and can move up or down. If at any point, Philip is in the same cell with a train, he loses. If the train reaches the left column, it continues to move as before, leaving the tunnel.

Your task is to answer the question whether there is a sequence of moves such that Philip would be able to get to the rightmost column.

\textbf{Input} \newline
The first line contains two integers $n, k$ $(2 \leq  n  \leq 100, 1 \leq k \leq 26)$ — the number of columns on the field and the number of trains. Each of the following three lines contains $n$ characters, representing the state of that row. Philip's initial position is marked as 's', and is in the leftmost column. Each of the $k$ trains is marked by a distinct string of uppercase letters of the English alphabet, located in one line. Character '.' represents an empty cell.


\textbf{Output} \newline
On a single print ``YES'' if it is possible for Philip to reach the end safely, and ``NO'' if it is not possible.

\textbf{Examples}
\begin{itemize}
\item \textbf{Input} \newline
16 4 \newline
...AAAAA........ \newline
s.BBB......CCCCC \newline
........DDDDD...

\textbf{Output} \newline
YES

\item \textbf{Input} \newline
16 4 \newline
...AAAAA........ \newline
s.BBB....CCCCC.. \newline
.......DDDDD....

\textbf{Output} \newline
NO

\end{itemize}


\end{document}